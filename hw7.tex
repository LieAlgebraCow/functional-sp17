\documentclass[a4paper,10pt]{report}
\usepackage[utf8]{inputenc}
usepackage{amsmath}
\usepackage{amssymb}
\usepackage{amsthm}
\usepackage{mathtools}
\usepackage{fancyhdr}
\usepackage{enumitem}
\usepackage[top=1in,left=1in,right=1in]{geometry}
\usepackage{mathrsfs}
\usepackage{bm}

\setenumerate{listparindent=\parindent,topsep=\parskip}
% \setlist[enumerate]{topsep=\parskip}
\setlist[enumerate,2]{label=(\arabic*)}
\setlist[enumerate,3]{label=(\alph*)}

% \newcommand{\set}[1]{{\{#1\}}}
\newcommand{\ggen}[1]{\langle#1\rangle}
\newcommand{\pn}[2]{||#1||_{#2}}
\newcommand{\bpn}[2]{\left|\left|#1\right|\right|_{#2}}
\newcommand{\norm}[1]{||#1||}
\newcommand{\bnorm}[1]{\left|\left|#1\right|\right|}
\DeclarePairedDelimiter{\ceil}{\lceil}{\rceil}
\DeclarePairedDelimiter{\floor}{\lfloor}{\rfloor}
\DeclarePairedDelimiter{\set}{\{}{\}}
\DeclarePairedDelimiter{\abs}{|}{|}
\DeclarePairedDelimiter{\ket}{|}{\rangle}
\DeclarePairedDelimiter{\bra}{\langle}{|}

\newcommand{\op}[1]{\mathring{#1}}
\newcommand{\cl}[1]{\overline{#1}}
\newcommand{\ol}[1]{\overline{#1}}

\renewcommand{\mod}{\text{ mod }}

\renewcommand{\O}{\operatorname{O}} % Bound otherwise
\renewcommand{\o}{\operatorname{o}}
\newcommand{\T}{\text{yes}}
\newcommand{\F}{\text{no}}

\newcommand{\Z}{\mathbb{Z}}
\newcommand{\R}{\mathbb{R}}
\newcommand{\N}{\mathbb{N}}
\newcommand{\C}{\mathbb{C}}
\newcommand{\Q}{\mathbb{Q}}
\newcommand{\Prob}{\mathbb{P}}
\newcommand{\E}{\mathbb{E}}
\newcommand{\Ba}{\ensuremath Ba}
\newcommand{\del}{\partial}
\newcommand{\tens}{\otimes}
\newcommand{\rar}[2][]{\overset{#2}{\underset{#1}{\longrightarrow}}}

\DeclareMathOperator{\img}{img}
\DeclareMathOperator{\sgn}{sgn}
\DeclareMathOperator{\fop}{int}
\DeclareMathOperator{\fcl}{cl}
% \DeclareMathOperator{\lg}{lg}
\DeclareMathOperator{\vspan}{span}
\DeclareMathOperator{\diam}{diam}
\DeclareMathOperator{\id}{id}
% \DeclareMathOperator{\Im}{Im}
% \DeclareMathOperator{\Rp}{Rp}
\DeclareMathOperator{\rng}{rng}
\DeclareMathOperator{\Spec}{spec}
\DeclareMathOperator{\Rng}{Rng}
\DeclareMathOperator{\Cov}{Cov}
\DeclareMathOperator{\Var}{Var}
\DeclareMathOperator{\Bernoulli}{Bernoulli}
\DeclareMathOperator{\Normal}{Normal}
\DeclareMathOperator{\Uniform}{Uniform}
\DeclareMathOperator{\Binom}{Binomial}
\DeclareMathOperator{\mgf}{mgf}
\DeclareMathOperator{\Supp}{Supp}
\DeclareMathOperator{\Tr}{Tr}
\DeclareMathOperator{\tr}{tr}

\newcommand{\cat}[1]{(\bm{#1})}

\providecommand{\Alpha}{A}

\newtheorem*{lemma*}{Lemma}

% Roman numerals
\makeatletter
\newcommand{\Romnum}[1]{\expandafter\@slowromancap\romannumeral #1@}
\makeatother
\newcommand{\factor}[1]{\text{\Romnum{#1}}}

\begin{document}
% \maketitle

\pagestyle{fancy}	
\fancyhf{} % Reset headers and footers
\lhead{Ethan Ackelsberg, Zachery Dell, Peter Huston\\
Functional Analysis 2\\
\today}
\setlength{\headheight}{60pt}

\begin{center}
	\textbf{Homework 7}
\end{center}

\begin{enumerate}
		\setcounter{enumi}{83}
 \item 
		\begin{enumerate}
			\item Fix $n\ge 2$. We may naturally identify $S_n\cong\set{\phi\in S_n:\phi(m)=m\text{ for all }m>n}\subseteq S_\infty$. There are infinitely many subgroups conjugate to $S_n$ in $S_\infty$ with trivial intersection: if $\phi_{n,k}=\prod_{j=1}^n(j,kn+j)$, then $\phi_{n,k}S_n\phi_{n,k}\bigcap\phi_{n,i}S_n\phi_{n,i}$ is trivial whenever $i\neq j$. Since every $\phi\in S_\infty$ lies in some $S_n$, if $\phi$ is not the identity, there are infinitely many distinct conjugates of $\phi$ in $S_\infty$. Therefore, $S_\infty$ is ICC. We proved in class that this makes $LS_\infty$ a type $\factor{2}_1$ factor. 
			\item Let an integer $m=n^k$ be given. Let $H$ be a subgroup of $S_\infty$ of order $m$, for example generated by a cycle of length $m$. Set $p=m^{-1}\sum_{g\in H}\lambda_g$. Since $\lambda_{g^{-1}}=\lambda_g^*$, this $p$ is self-adjoint. Also, $p^2=m^{-2}\sum_{(g,h)\in H\times H}\lambda_g\lambda_h=m^{-1}\sum_{k\in H}\lambda_k=p$. Therefore, $p\in P(L\Gamma)$. Of course, $\tr(p)=\ggen{p\delta_e,\delta_e}=m^{-1}\sum_{g\in H}\ggen{\delta_g,\delta_e}=m^{-1}$, as desired. 
			\item Since $S_\infty=\bigcup_{n\in\N}S_n$, we have $\C S_\infty=\bigcup_{n\in\N}\C S_n$, and $LS_\infty=(\C S_\infty)''$ by definition. 
		\end{enumerate}
		\setcounter{enumi}{85}
 \item 

		Finally, consider the case that $M$ is a purely infinite factor, that is, type $\factor{3}$, and suppose that $M$ is countably decomposable. In this case, we adapt a proof from Jesse Peterson's notes on Von Neumann algebras. Let $p\in P(M)\setminus\set{0}$ be a non-zero projection. We will show that $p\approx 1$. First, we find $q\le p$ with $q\approx 1-q\approx p$. Since $p$ must be infinite, there is some partial isometry $u\in M$ with $u^*u=p$ and $uu^*<p$ yet $uu^*\approx p$. Define $p_0=1-uu^*$, and for $n\in\N$, define $p_n=u^np_0u^*$. Then $p_n$ is a projection, and since $uu^*\le p$ and $u^*u\le p$, we have $p_n\le p$. 
		
		For $n>m$, we have 
		\begin{align*}
			p_np_m &= (u^n(1-uu^*)(u^*)^n)(u^m(1-uu^*)(u^*)^m)\\
			&= u^n(u^*)^nu^m(u^*)^m-u^n(u^*)^nu^{m+1}(u^*)^{m+1}-u^{n+1}(u^*)^{n+1}u^m(u^*)^m+u^{n+1}(u^*)^{n+1}u^{m+1}(u^*)^{m+1}\\
			\intertext{Since $uu^*\le p$ and $u^*u\le p$, the presence of $p=u^*u$ in the above terms is redundant. Cancelling $m+1$ times gives }
			p_np_m &= u^n(u^*)^{n-m}u^*m-u^n(u^*){n-(m+1)}(u^*)^{m+1}-u^{n+1}(u^*)^{(n+1)-m}(u^*)^m+u^{n+1}(u^*)^{n-m}(u^*)^{m+1}\\
			&= 0
		\end{align*}
		Also notice that just as $up_nu^*=p_{n+1}$, we have $u^*p_nu=(u^*u)u^{n-1}p_0(u^*)^{n-1}(u^*u)=p_{n-1}$; for all $n$ and $m$, we have $p_n\approx p_m$. Therefore, $(p_n)_{n\in\N}$ is a sequence of mutually orthogonal projections, each equivalent to $p_0$ and bounded above by $p$. 

		By Zorn's lemma, we may extend $(p_n)_{n\in\omega}$ to a maximal family of mutually orthogonal equivalent projections bounded above by $p$, say $(q_n)_{n\in\omega}$; we may assume the family is countable since $M$ is countably decomposable. Let $q_\omega=1-\sum_{n\in\omega}q_n$. If $q_\omega\succ q_0$, then there exists $v$ a partial isometry with $v^*v\le q_\omega$ and $vv^*\approx p_0$, contradicting the maximality of $(q_n)_{n\in\omega}$. Therefore, $q_\omega\prec q_0$. But we know that if $(a_i)$ and $(b_i)$ are families of mutually orthogonal projections with $a_i\prec b_i$ for each $i$, then $\sum_ia_i\prec\sum_ib_i$. % HERE why
		In particular, picking some bijection between $\omega+1$ and $\omega$, we have 
		\[p=\sum_{n\in(\omega+1)}q_n\prec\sum_{n\in\omega}q_n\le p\], and hence $\sum_{n\in\omega}q_n\approx p$. Similarly, $p\approx\sum_{n\in\omega}q_{2n}\approx\sum_{n\in\omega}q_{2n+1}$, so letting $q=\sum_{n\in\omega}q_{2n}$, we have $q\le p$ and $q,p-q\approx p$. 

		By repeating the above with $p_0=q$, we may as well assume that $p_0\approx p-p_0\approx p$ in the first place. Further extend $(q_n)_{n\in\omega}$ to a maximal family of mutually orthogonal projections $(r_n)_{n\in\omega}$ with each $r_n\prec p$; this family is still countable, again by countable decomposability of $M$. If $1-\sum_{n\in\omega}r_n\neq 0$, then just as before, we can find $r_\omega\le 1-\sum_{n\in\omega}r_n$ such that $r_\omega\prec p$, contradicting maximality. Therefore, $1=\sum_{n\in\omega}r_n$. Since each $p_n\approx p_0\approx p$, we have 
		\[1=\sum_{n\in\omega}r_n\prec\sum_{n\in\omega}p_n\le p\]
	 so we have $1\approx p$. 

  This means that there is a partial isometry $u$ with $u^*u=1$ and $uu^*=p$, so that $1=u^*(uu^*)u$ is conjugate to $p$ in $M$. If $I\subseteq M$ is a non-zero two-sided ideal, then by a previous application of the spectral theorem, % HERE was that why? Say more?
		there is a non-zero projection $p\in I$, and consequently, $1\in I$ and $I=M$.
 \item 
 \item 
\end{enumerate}

\end{document}          
